\hypertarget{software-development-processsdp}{%
\section{Software Development
Process(SDP)}\label{software-development-processsdp}}

\textbf{Authors:} Jake Goodwin, Aidan Agee, Blake Babb, Patrick Iacob

\textbf{DATE:} 2023-11-16

\hypertarget{principles}{%
\section{Principles}\label{principles}}

\begin{itemize}
\tightlist
\item
  We will respond to asynchronous communication within 24 hours
\item
  We will be at meetings on time and pay attention
\item
  All changes need to be isolated to their own git branch
\item
  Each work item need a corresponding GitHub issue
\item
  Pull Requests have to be reviewed by at least one team member
\item
  We will use a kanban board to continuously work on the backlog
\item
  Once a work item is complete, a pull request is created
\item
  Blocks need to be discussed as soon as possible
\end{itemize}

\hypertarget{process}{%
\section{Process}\label{process}}

Task Selection: * Kanban style * Select highest priority item within
your role's domain

To Solve an Issue and Meet Acceptance Criteria: 1. Write failing tests.
2. Write code to pass tests. 3. Repeat. 4. Open a pull-request and merge

\hypertarget{steps-in-software-development}{%
\subsection{Steps in software
development}\label{steps-in-software-development}}

\begin{enumerate}
\def\labelenumi{\arabic{enumi}.}
\tightlist
\item
  Create a github issue/milestone.
\item
  Assign github task/issue.
\item
  Create fork of repo.
\item
  Write tests using the test framework.
\item
  Push the tests to the fork (optional).
\item
  Write Code to pass the tests.
\item
  Test the code.
\item
  Push to the fork repo.
\item
  Create a pull request with description.
\item
  Get approval for the PR(pull request).
\end{enumerate}

\hypertarget{goals-objectives}{%
\subsection{Goals \& Objectives}\label{goals-objectives}}

Develop a personal data acquisition system that records all the data a
user might want, and is cheap and easy to set up and use. * Record data
on acceleration, force, position, etc. * Minimal setup * Can be hooked
up to bike, go-kart, etc.

\hypertarget{project-scope}{%
\subsection{Project Scope}\label{project-scope}}

\begin{itemize}
\tightlist
\item
  Design of simple UI to display data.
\item
  Design of hardware/schematics for system.
\item
  Firmware for sensor modules in rust.
\item
  SBC with rust software to store/log sensor data over CAN.
\end{itemize}

\hypertarget{roles}{%
\section{Roles}\label{roles}}

\begin{longtable}[]{@{}lll@{}}
\toprule
ROLE & PERSON & RESPONSIBILITIES\tabularnewline
\midrule
\endhead
UI & Blake & Develops the web page front end\tabularnewline
SBC/SW & Aidian & Develop logic to relay sensor data to
UI\tabularnewline
FIRMWARE & Patrick & Develop firmware for
microcontrollers\tabularnewline
HARDWARE & Jake & Design schematics, wiring diagrams \& PCB
files\tabularnewline
\bottomrule
\end{longtable}

These are the general outlines for the four different roles in the
project. We have a verbal agreement at the moment that we will help out
with parts of the project outside our roles as needed.

\hypertarget{tooling}{%
\section{Tooling}\label{tooling}}

\begin{longtable}[]{@{}ll@{}}
\toprule
Purpose & Name\tabularnewline
\midrule
\endhead
Version Control & Git\tabularnewline
Project Management & GitHub Projects\tabularnewline
Documentation & Rustdocs \& MD\tabularnewline
Test framework & Rust \& Cmocka\tabularnewline
Editor & ANY\tabularnewline
Schematics \& PCB & KiCAD\tabularnewline
Communication & Discord/Teams/Email\tabularnewline
&\tabularnewline
\bottomrule
\end{longtable}

\hypertarget{version-control}{%
\subsection{Version Control}\label{version-control}}

Git will allow our team to track changes in the projects files over
time. Also prevents the loss of work from hardware failures.

\hypertarget{project-management}{%
\subsection{Project Management}\label{project-management}}

GitHub projects is integrated into github organizations as well as git.
The project management software makes the collaboration between
developers easy and will make tracking milestones and issues for the
entire project across multiple repositories a possibility.

\hypertarget{documentation}{%
\subsection{Documentation}\label{documentation}}

Documentation will primaily be done through the built-in rust-docs
feature. This is accesiable via the CLI(command line interface) tooling.
This will encapsulate how the code itself and any interfaces are
documented.

Because the documentation is genreated as part of the code this will
ensure that up to date and accurate documentatio is always availble.

Secondary documentation meant for non-developers will be done using a
combination of markdown and LaTex where needed. This will be availble
usally in a PDF format.

\hypertarget{schematics-pcb}{%
\subsection{Schematics \& PCB}\label{schematics-pcb}}

The KiCAD program gives access to the schematics and PCB designs to all
team members due to the software begin free and open-source.

It will allow us to comment, label and design the needed circuits for
the physical hardware of the system; providing a good troubleshooting
resource as well.

\hypertarget{communication}{%
\subsection{Communication}\label{communication}}

\textbf{Discord:} * Used to coordinate team meetings. * To share
ideas/brainstorm * give updates on project.

\textbf{Teams:} * TA meetings.

\textbf{GITHUB:} * To discuss project issues. * share documentation.

\hypertarget{definition-of-donedod}{%
\section{Definition of Done(DOD)}\label{definition-of-donedod}}

\begin{itemize}
\tightlist
\item
  Acceptance criteria all satisfied by code changes
\item
  Changes have been merged to master after completing the Pull Request
  Process
\item
  A completed Pull Request has at least one approval and no marks for
  ``Needs Work''
\item
  All tests pass with changes implemented and no reversion is required
\item
  Relevant documentation for the feature has been updated
\item
  Discussion points are prepared for next meeting
\end{itemize}

\hypertarget{testing}{%
\subsection{TESTING}\label{testing}}

\textbf{Rust:}

The testing for all code repositories will be done using a testing
harness or framework. For rust this takes the form of the
\texttt{cargo\ test} command, which is part of the package managment
system(tool-chain).

These tests will be used as one of acceptance critera for a branch to be
pulled into the main branch.

\textbf{C:}

Some libraries or areas where the use of C code is needed we plan to use
cmocka as the unit testing framework. This combined with Cmake as the
build system will give us a host agnostic development cycle.

\hypertarget{quality-assurance}{%
\subsubsection{Quality Assurance}\label{quality-assurance}}

Quality assurance will mostly be handled by adhearance to style
standards enforced by the lanuages LSP(language server protocol)
servers. The two that will see extensive use in this project being:

\begin{enumerate}
\def\labelenumi{\arabic{enumi}.}
\tightlist
\item
  Rust-analyzer
\item
  clangd
\end{enumerate}

\hypertarget{feedback}{%
\subsubsection{Feedback}\label{feedback}}

Feedback on the work done will take place in the github projects. The
issues and discussion boards are the main locations for this, with the
weekly meetings and discord being a secondary and informal medium for
minor feedback.

\hypertarget{release-cycle}{%
\section{Release Cycle}\label{release-cycle}}

For the moment we will used semantic versioning with the standard
Major.minor.patch format. This will help when it comes to dealing with
any major changes that break APIs.

\hypertarget{contingency-plans}{%
\subsection{Contingency Plans}\label{contingency-plans}}

Feedback can be shared during weekly standup with the TA, or over
Discord if they are more time-sensitive, after which it should be
reviewed by the whole team, and then incorporated. Changes to the whole
process will require more comprehensive feedback and approval from the
team before going into effect, after which related documents should be
modified as soon as possible.

In the event of unexpected challenges, the team should be notified
immediately, and if serious enough should be brought up with the TA,
project partner or instructor, otherwise they should be brought up
during regular meetings.

\hypertarget{timeline}{%
\section{Timeline}\label{timeline}}

\begin{itemize}
\tightlist
\item
  12/15/2023: Version 0 complete with breadboard organized hardware and
  visual UI elements
\item
  03/22/2023: Version 1 complete with functionality between firmware,
  SBC, and UI
\end{itemize}

\hypertarget{environments}{%
\section{Environments}\label{environments}}

\begin{longtable}[]{@{}lllll@{}}
\toprule
\begin{minipage}[b]{0.11\columnwidth}\raggedright
Environment\strut
\end{minipage} & \begin{minipage}[b]{0.21\columnwidth}\raggedright
Infrastructure\strut
\end{minipage} & \begin{minipage}[b]{0.22\columnwidth}\raggedright
Deployment\strut
\end{minipage} & \begin{minipage}[b]{0.23\columnwidth}\raggedright
What is it for?\strut
\end{minipage} & \begin{minipage}[b]{0.10\columnwidth}\raggedright
Monitoring\strut
\end{minipage}\tabularnewline
\midrule
\endhead
\begin{minipage}[t]{0.11\columnwidth}\raggedright
Production\strut
\end{minipage} & \begin{minipage}[t]{0.21\columnwidth}\raggedright
Github releases\strut
\end{minipage} & \begin{minipage}[t]{0.22\columnwidth}\raggedright
Release\strut
\end{minipage} & \begin{minipage}[t]{0.23\columnwidth}\raggedright
Packaging install files.\strut
\end{minipage} & \begin{minipage}[t]{0.10\columnwidth}\raggedright
N/A\strut
\end{minipage}\tabularnewline
\begin{minipage}[t]{0.11\columnwidth}\raggedright
Staging\strut
\end{minipage} & \begin{minipage}[t]{0.21\columnwidth}\raggedright
Github actions\strut
\end{minipage} & \begin{minipage}[t]{0.22\columnwidth}\raggedright
\strut
\end{minipage} & \begin{minipage}[t]{0.23\columnwidth}\raggedright
\strut
\end{minipage} & \begin{minipage}[t]{0.10\columnwidth}\raggedright
Github Pull requests\strut
\end{minipage}\tabularnewline
\begin{minipage}[t]{0.11\columnwidth}\raggedright
Development\strut
\end{minipage} & \begin{minipage}[t]{0.21\columnwidth}\raggedright
Local\strut
\end{minipage} & \begin{minipage}[t]{0.22\columnwidth}\raggedright
Github commits\strut
\end{minipage} & \begin{minipage}[t]{0.23\columnwidth}\raggedright
Development and unit tests of microcontroller-based sensors\strut
\end{minipage} & \begin{minipage}[t]{0.10\columnwidth}\raggedright
Manual\strut
\end{minipage}\tabularnewline
\bottomrule
\end{longtable}
